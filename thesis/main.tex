% !BIB TS-program = none
%%%%%%%%%%%%%%%%%%%%%%%%%%%%%%%%%%%%%%%%%12pt: font size
                                        %a4paper: A4 format
                                        %openright: open chapters on right pages
                                        %twoside: for two-sided printing
                                        %report: thesis/book style
\documentclass[12pt,a4paper,openright,twoside]{book}

%%%%%%%%%%%%%%%%%%%%%%%%%%%%%%%%%%%%%%%%% language package
\usepackage[english]{babel}

%%%%%%%%%%%%%%%%%%%%%%%%%%%%%%%%%%%%%%%%% input encoding
\usepackage[latin1]{inputenc}

%%%%%%%%%%%%%%%%%%%%%%%%%%%%%%%%%%%%%%%%% layout
\usepackage{fancyhdr}
\usepackage{indentfirst}
\usepackage{graphicx}
\usepackage{newlfont}

%%%%%%%%%%%%%%%%%%%%%%%%%%%%%%%%%%%%%%%%% math packages
\usepackage{amssymb}
\usepackage{amsmath}
\usepackage{latexsym}
\usepackage{amsthm}

\oddsidemargin=30pt 
\evensidemargin=20pt

\usepackage{graphicx}
% Make image paths work both when compiling main.tex and individual subfiles.
\graphicspath{{\subfix{images/}}}

\usepackage{csquotes}
\usepackage{tcolorbox}
\usepackage{booktabs} 
\usepackage{booktabs}
\usepackage{array} % for p{} column type
\usepackage{geometry}
\usepackage{subfiles}
\usepackage[numbers]{natbib}
\usepackage[hidelinks]{hyperref}
\usepackage{cleveref}
\usepackage{xcolor}
\definecolor{darkred}{RGB}{139,0,0}
\usepackage{listings}
\lstset{
    breaklines=true,
    basicstyle=\ttfamily\color{darkred},
    keywordstyle=\color{darkred},
    commentstyle=\color{darkred},
    stringstyle=\color{darkred},
    % postbreak=\mbox{\color{darkred}{$\hookrightarrow$}\space}, % Optional: shows where the break occurred
}
\lstdefinestyle{featstyle}{
    columns=fullflexible,
    keepspaces=true,
    showstringspaces=false
}
\usepackage{float}
\usepackage{subcaption}
\newtcolorbox{definition}[1]{
        colback=gray!10,
    colframe=gray,
    boxrule=0.8pt,
    arc=2mm,
    left=6pt,
    right=6pt,
    top=6pt,
    bottom=6pt,
    title={#1},
    coltitle=white,
    colbacktitle=gray
}


\geometry{margin=2cm} 

\hyphenation{} % empty or add English words if needed

\pagestyle{fancy}\addtolength{\headwidth}{20pt}
\renewcommand{\chaptermark}[1]{\markboth{\thechapter.\ #1}{}}
\renewcommand{\sectionmark}[1]{\markright{\thesection \ #1}{}}
\rhead[\fancyplain{}{\bfseries\leftmark}]{\fancyplain{}{\bfseries\thepage}}
\cfoot{}
\linespread{1.3}

\begin{document}
\setlength{\headheight}{14.49998pt}
\addtolength{\topmargin}{-2.49998pt}

\begin{titlepage}
\thispagestyle{empty}
\topmargin=6.5cm
\raggedleft
\large
This is the \textsc{Dedication}:\\
you can write whatever you want here, \\
or nothing at all \ldots
\newpage
\clearpage{\pagestyle{empty}\cleardoublepage}
\end{titlepage}

\pagenumbering{roman}
\chapter*{Introduction}
\rhead[\fancyplain{}{\bfseries INTRODUCTION}]{\fancyplain{}{\bfseries\thepage}}
\lhead[\fancyplain{}{\bfseries\thepage}]{\fancyplain{}{\bfseries INTRODUCTION}}
\addcontentsline{toc}{chapter}{Introduction}

\subfile{sections/Intro/intro.tex}

\clearpage{\pagestyle{empty}\cleardoublepage}
\tableofcontents

% \clearpage{\pagestyle{empty}\cleardoublepage}
% \listoffigures

% \clearpage{\pagestyle{empty}\cleardoublepage}
% \listoftables

\clearpage{\pagestyle{empty}\cleardoublepage}

\chapter{Technical Background}
\label{chap:techBg}
\rhead[\fancyplain{}{\bfseries TECHNICAL BACKGROUND}]{\fancyplain{}{\bfseries\thepage}}
\lhead[\fancyplain{}{\bfseries\thepage}]{\fancyplain{}{\bfseries\rightmark}}
\pagenumbering{arabic}
\subfile{sections/background/background.tex}

\chapter{Methodology}
\label{chap:methodology}
\rhead[\fancyplain{}{\bfseries METHODOLOGY}]{\fancyplain{}{\bfseries\thepage}}
\lhead[\fancyplain{}{\bfseries\thepage}]{\fancyplain{}{\bfseries\rightmark}}
\subfile{sections/c1/c1.tex}

\chapter{Experimental Evaluation}
\label{chap:expEval}
\rhead[\fancyplain{}{\bfseries EXPERIMENTAL EVALUATION}]{\fancyplain{}{\bfseries\thepage}}
\lhead[\fancyplain{}{\bfseries\thepage}]{\fancyplain{}{\bfseries\rightmark}}
\subfile{sections/experiments/experiments.tex}

\chapter{A FlatZinc Parser: fzn2nl}
\label{chap:parser}
\rhead[\fancyplain{}{\bfseries FLATZINC PARSER}]{\fancyplain{}{\bfseries\thepage}}
\lhead[\fancyplain{}{\bfseries\thepage}]{\fancyplain{}{\bfseries\rightmark}}
\subfile{sections/parserDesc/parserDesc.tex}

\chapter*{Appendix}
\rhead[\fancyplain{}{\bfseries APPENDIX}]{\fancyplain{}{\bfseries\thepage}}
\lhead[\fancyplain{}{\bfseries\thepage}]{\fancyplain{}{\bfseries APPENDIX}}
\addcontentsline{toc}{chapter}{Appendix}

\subfile{sections/appendix/appendix.tex}

\clearpage{\pagestyle{empty}\cleardoublepage}
\begin{thebibliography}{90}             
\rhead[\fancyplain{}{\bfseries BIBLIOGRAPHY}]{\fancyplain{}{\bfseries\thepage}}
\lhead[\fancyplain{}{\bfseries\thepage}]{\fancyplain{}{\bfseries BIBLIOGRAPHY}}


\addcontentsline{toc}{chapter}{Bibliography}
%%%%%%% Technical Background %%%%%%%
\bibitem{CPholygrail} R. Bart$\acute{a}$k, ``Constraint Programming: In Pursuit of the Holy Grail'', Jan. 1999.
\bibitem{holygrail} E. C. Freuder, ``In Pursuit of the Holy Grail'', Constraints, vol. 2, no. 1, pp. 57-61, Apr. 1997, doi: https://doi.org/10.1023/a:1009749006768.
\bibitem{MiniZinc}N. Nethercote, P. J. Stuckey, R. Becket, S. Brand, G. J. Duck, and G. Tack, ``MiniZinc: Towards a Standard CP Modelling Language'', Springer eBooks, pp. 529-543, Oct. 2007, doi: https://doi.org/10.1007/978-3-540-74970-7\_38.
\bibitem{SeaPearl} F. Chalumeau, I. Coulon, Q. Cappart, and L.-M. Rousseau, ``SeaPearl: A Constraint Programming Solver guided by Reinforcement Learning'', arXiv.org, Apr. 20, 2021. https://arxiv.org/abs/2102.09193
\bibitem{PortfolioSolvers} R. Amadini, M. Gabbrielli, and J. Mauro, ``Why CP Portfolio Solvers Are (under)Utilized? Issues and Challenges'', Lecture Notes in Computer Science, pp. 349-364, 2015, doi: \url{https://doi.org/10.1007/978-3-319-27436-2_21}.
\bibitem{AlgPortfolio} Gomes, C.P., Selman, B.: ``Algorithm portfolios''. Artif. Intell. 126(1-2), 43-62 (2001)
\bibitem{AlgSelProb} Rice, J.R.: ``The algorithm selection problem''. Adv. Comput. 15, 65-118 (1976)
\bibitem{AlgSelCSP} Kotthoff, L.: ``Algorithm selection for combinatorial search problems: a survey''. AI Mag. 35(3), 48-60 (2014)
\bibitem{fromPFStoAS} Roberto Amadini, Simone Gazza. ``From Portfolio Solvers to Agentic Solvers''. LLM-Solve, Tias Guns; Serdar Kadioglu; Stefan Szeider; Dimos Tsouros, Aug 2025, Glasgow, United Kingdom, United Kingdom. pp.569 - 585, 10.5281/zenodo.17640236. hal-05446738
\bibitem{LLMsIBM} IBM, ``What are large language models (LLMs)?'', Ibm.com, Nov. 02, 2023. https://www.ibm.com/think/topics/large-language-models
\bibitem{TransformerArchitecture} A. Vaswani et al., ``Attention Is All You Need'', arXiv.org, 2017. https://arxiv.org/abs/1706.03762
\bibitem{InfeerenceTrainingLLMs} Y. Liu et al., ``Understanding LLMs: A Comprehensive Overview from Training to Inference'', arXiv (Cornell University), Jan. 2024, doi: https://doi.org/10.48550/arxiv.2401.02038.
\bibitem{LLMDecoding} C. Shi et al., ``A Thorough Examination of Decoding Methods in the Era of LLMs'', arXiv (Cornell University), Feb. 2024, doi: https://doi.org/10.48550/arxiv.2402.06925.
\bibitem{BeamSearch} Markus Freitag and Yaser Al-Onaizan. 2017. ``Beam search strategies for neural machine translation. In Proceedings of the First Workshop on Neural Machine Translation.''
\bibitem{ToppSampling}Ari Holtzman, Jan Buys, Li Du, Maxwell Forbes, and Yejin Choi. 2020. ``The curious case of neural text degeneration. In 8th International Conference on Learning Representations'', ICLR 2020, Addis Ababa, Ethiopia, April 26-30, 2020.
\bibitem{prompting} P. Sahoo, A. K. Singh, S. Saha, V. Jain, S. Mondal, and A. Chadha, ``A Systematic Survey of Prompt Engineering in Large Language Models: Techniques and Applications'', arXiv (Cornell University), Feb. 2024, doi: https://doi.org/10.48550/arxiv.2402.07927.
\bibitem{ZeroFewShot} S. Rahman, S. Khan, and F. Porikli, ``A Unified Approach for Conventional Zero-Shot, Generalized Zero-Shot, and Few-Shot Learning'', IEEE Transactions on Image Processing, vol. 27, no. 11, pp. 5652-5667, Nov. 2018, doi: https://doi.org/10.1109/tip.2018.2861573.
\bibitem{APIS} Farhan Nadim Iqbal, ``A BRIEF INTRODUCTION TO APPLICATION PROGRAMMING INTERFACE (API)'', zenodo.org, doi: https://doi.org/10.5281/zenodo.10198423.
%%%%%% Experimental Evaluation %%%%%%%
\bibitem{GeminiAPI} ``Gemini API Docs and Reference'', Google AI for Developers. https://ai.google.dev/gemini-api/docs (accessed Dec. 10, 2025). 
\bibitem{GeminiAPIver} ``API versions explained'', Google AI for Developers, 2025. https://ai.google.dev/gemini-api/docs/api-versions (accessed Dec. 11, 2025).
\bibitem{GroqAPI} ``GroqCloud'', Groq.com, 2024. https://console.groq.com/docs/overview (accessed Dec. 10, 2025).
\bibitem{GroqRL} ``Rate Limits - GroqDocs'', GroqDocs, 2025. https://console.groq.com/docs/rate-limits (accessed Dec. 10, 2025).
\bibitem{GeminiRL} ``Rate limits'', Google AI for Developers, 2025. https://ai.google.dev/gemini-api/docs/rate-limits (accessed Dec. 10, 2025).
\bibitem{ContextRot} Kelly Hong, Anton Troynikov, Jeff Huber, ``Context Rot: How Increasing Input Tokens Impacts LLM Performance'', Trychroma.com, 2025. https://research.trychroma.com/context-rot?ref=blog.promptlayer.com (accessed Dec. 11, 2025).
\bibitem{MznProblems} ``MiniZinc - List of Problems and Globals used in the MiniZinc Challenge'', Minizinc.org, 2025. https://www.minizinc.org/challenge/globals/ (accessed Dec. 11, 2025).
\bibitem{MznResults25} ``MiniZinc - Challenge 2025 Results'', Minizinc.org, 2025. https://www.minizinc.org/challenge/2025/results/ (accessed Dec. 11, 2025).
\bibitem{PhilMznChallenge} P. J. Stuckey, T. Feydy, A. Schutt, G. Tack, and J. Fischer, ``The MiniZinc Challenge 2008-2013'', AI Magazine, vol. 35, no. 2, p. 55, Jun. 2014, doi: https://doi.org/10.1609/aimag.v35i2.2539.
\bibitem{evaluationMetaSolvers} R. Amadini, Maurizio Gabbrielli, T. Liu, and J. Mauro, ``On the Evaluation of (Meta-)solver Approaches'', Journal of Artificial Intelligence Research, vol. 76, pp. 705-719, Mar. 2023, doi: https://doi.org/10.1613/jair.1.14102.
\bibitem{PortfolioAproaches} R. Amadini, Maurizio Gabbrielli, and J. Mauro, ``Portfolio Approaches for Constraint Optimization Problems'', Lecture notes in computer science, pp. 21-35, Jan. 2014, doi: https://doi.org/10.1007/978-3-319-09584-4\_3.
\bibitem{lindauer2019} Lindauer, M., van Rijn, J. N., \& Kotthoff, L. (2019). ``The algorithm selection competitions'' 2015 and 2017.Artificial Intelligence,272, 86-100.
\bibitem{LLMdistract} Shi, F., Chen, X., Misra, K., Scales, N., Dohan, D., Chi, E., Sch{\"a}rli, N., and Zhou, D. (2023). ``Large Language Models Can Be Easily Distracted by Irrelevant Context'' arXiv preprint arXiv:2302.00093. doi: https://doi.org/10.48550/arXiv.2302.00093
\bibitem{RBFormatting}Sumanth P, ``Agentic Prompt Engineering: A Deep Dive into LLM Roles and Role-Based Formatting'', Clarifai.com, Jul. 2025. https://www.clarifai.com/blog/agentic-prompt-engineering\#title\_1 (accessed Jan. 03, 2026).
\bibitem{LLMhallucination} X. Lin et al., ``LLM-based Agents Suffer from Hallucinations: A Survey of Taxonomy, Methods, and Directions'', arXiv (Cornell University), Sep. 2025, doi: https://doi.org/10.48550/arxiv.2509.18970.
\bibitem{mzn2feat} R. Amadini, M. Gabbrielli, and J. Mauro. ``An Enhanced Features Extractor for a Portfolio of Constraint Solvers''. In SAC, 2014.
\bibitem{Gecode} M. Morara, J. Mauro, and M. Gabbrielli, ``Solving XCSP problems by using Gecode'', arXiv (Cornell University), Dec. 2011, doi: https://doi.org/10.48550/arxiv.1112.6096.
\bibitem{FlexBison} D. Guo, ``Development of Compiler Based on Flex and Bison'', Ordnance Industry Automation, Jan. 2004.
\bibitem{splitAndBounds} R. Amadini, and Peter J. Stuckey. ``Sequential Time Splitting and Bounds Communication for a Portfolio of Optimization Solvers''. In CP, 2014.
\bibitem{SUNNY-CP} R. Amadini, M. Gabbrielli, and J. Mauro. ``SUNNY-CP: a Sequential CP Portfolio Solver''. In SAC, 2015.
\bibitem{multicoreCS} R. Amadini, M. Gabbrielli, and J. Mauro. ``A Multicore Tool for Constraint Solving''. In IJCAI, 2015.
\bibitem{TempDef} D. H. Ackley, G. E. Hinton, and T. J. Sejnowski, ``A learning algorithm for boltzmann machines'', Cognitive Science, vol. 9, no. 1, pp. 147-169, Jan. 1985, doi: https://doi.org/10.1016/S0364-0213(85)80012-4.
\bibitem{NNetKnowledge} G. Hinton, O. Vinyals, and J. Dean, ``Distilling the Knowledge in a Neural Network'', arXiv.org, Mar. 09, 2015. http://arxiv.org/abs/1503.02531
\bibitem{ContextualTemp} P.-H. Wang et al., ``Contextual Temperature for Language Modeling'', arXiv (Cornell University), Jan. 2020, doi: https://doi.org/10.48550/arxiv.2012.13575.
\bibitem{PromptingWithLLM} R. Wang, H. Wang, F. Mi, Y. Chen, R. Xu, and K.-F. Wong, ``Self-Critique Prompting with Large Language Models for Inductive Instructions'', arXiv.org, May 23, 2023. https://arxiv.org/abs/2305.13733 (accessed Mar. 05, 2024).
\bibitem{GroqAPIdoc} ``Prompt Basics - GroqDocs'', GroqDocs, 2026. https://console.groq.com/docs/prompting
%%%%%%%%%%% Parser %%%%%%%%%%%
\bibitem{MemorizationVSReasoning} A. O. Li and T. Goyal, ``Memorization vs. Reasoning: Updating LLMs with New Knowledge'', arXiv (Cornell University), Apr. 2025, doi: https://doi.org/10.48550/arxiv.2504.12523.
\bibitem{FTLLMallucination} Z. Gekhman et al., ``Does Fine-Tuning LLMs on New Knowledge Encourage Hallucinations?'', arXiv (Cornell University), May 2024, doi: https://doi.org/10.48550/arxiv.2405.05904.
\bibitem{knowledgeCutoffsLLM} J. Cheng, M. Marone, O. Weller, D. Lawrie, D. Khashabi, and V. Durme, ``Dated Data: Tracing Knowledge Cutoffs in Large Language Models'', arXiv (Cornell University), Mar. 2024, doi: https://doi.org/10.48550/arxiv.2403.12958.
\bibitem{LLMwStructData} X. Wu and K. Tsioutsiouliklis, ``Thinking with Knowledge Graphs: Enhancing LLM Reasoning Through Structured Data'', arXiv (Cornell University), Dec. 2024, doi: https://doi.org/10.48550/arxiv.2412.10654.
\bibitem{mznCompiler} MiniZinc, ``GitHub - MiniZinc/libminizinc: The MiniZinc compiler'', GitHub, Jan. 23, 2026. https://github.com/MiniZinc/libminizinc (accessed Feb. 04, 2026).
\bibitem{LISP} Allen, J.: ``Anatomy of LISP''. McGraw-Hill, Inc., New York (1978)
\bibitem{MiniZincwFunctions} P. J. Stuckey and G. Tack, ``MiniZinc with Functions'', Lecture Notes in Computer Science, pp. 268-283, 2013, doi: https://doi.org/10.1007/978-3-642-38171-3\_18.
\bibitem{Gurobi} ``Gurobi Optimizer'', Gurobi Optimization. https://www.gurobi.com/solutions/gurobi-optimizer/
\bibitem{CPvsMILPsolvers} R. M. e S. de Oliveira and M. S. F. O. de C. Ribeiro, ``Comparing Mixed \& Integer Programming vs. Constraint Programming by solving Job-Shop Scheduling Problems'', Independent Journal of Management \& Production, vol. 6, no. 1, Mar. 2015, doi: https://doi.org/10.14807/ijmp.v6i1.262.
\end{thebibliography}

\clearpage{\pagestyle{empty}\cleardoublepage}
\chapter*{Acknowledgements}
\thispagestyle{empty}
Here you can thank whoever you want.
\end{document}