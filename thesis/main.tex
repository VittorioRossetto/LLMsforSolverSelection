% !BIB TS-program = none
%%%%%%%%%%%%%%%%%%%%%%%%%%%%%%%%%%%%%%%%%12pt: font size
                                        %a4paper: A4 format
                                        %openright: open chapters on right pages
                                        %twoside: for two-sided printing
                                        %report: thesis/book style
\documentclass[12pt,a4paper,openright,twoside]{book}

%%%%%%%%%%%%%%%%%%%%%%%%%%%%%%%%%%%%%%%%% language package
\usepackage[english]{babel}

%%%%%%%%%%%%%%%%%%%%%%%%%%%%%%%%%%%%%%%%% input encoding
\usepackage[latin1]{inputenc}

%%%%%%%%%%%%%%%%%%%%%%%%%%%%%%%%%%%%%%%%% layout
\usepackage{fancyhdr}
\usepackage{indentfirst}
\usepackage{graphicx}
\usepackage{newlfont}

%%%%%%%%%%%%%%%%%%%%%%%%%%%%%%%%%%%%%%%%% math packages
\usepackage{amssymb}
\usepackage{amsmath}
\usepackage{latexsym}
\usepackage{amsthm}

\oddsidemargin=30pt 
\evensidemargin=20pt

\usepackage{graphicx}
\graphicspath{ {./images/} }

\usepackage{tcolorbox}
\usepackage{booktabs} 
\usepackage{booktabs}
\usepackage{array} % for p{} column type
\usepackage{geometry}
\usepackage{subfiles}
\usepackage[numbers]{natbib}
\usepackage{hyperref}
\usepackage{cleveref}
\usepackage{xcolor}
\definecolor{darkred}{RGB}{139,0,0}
\usepackage{listings}
\lstset{
    breaklines=true,
    basicstyle=\ttfamily\color{darkred},
    keywordstyle=\color{darkred},
    commentstyle=\color{darkred},
    stringstyle=\color{darkred},
    % postbreak=\mbox{\color{darkred}{$\hookrightarrow$}\space}, % Optional: shows where the break occurred
}
\usepackage{float}
\usepackage{subcaption}
\newtcolorbox{definition}[1]{
        colback=gray!10,
    colframe=gray,
    boxrule=0.8pt,
    arc=2mm,
    left=6pt,
    right=6pt,
    top=6pt,
    bottom=6pt,
    title={#1},
    coltitle=white,
    colbacktitle=gray
}


\geometry{margin=2cm} 

\hyphenation{} % empty or add English words if needed

\pagestyle{fancy}\addtolength{\headwidth}{20pt}
\renewcommand{\chaptermark}[1]{\markboth{\thechapter.\ #1}{}}
\renewcommand{\sectionmark}[1]{\markright{\thesection \ #1}{}}
\rhead[\fancyplain{}{\bfseries\leftmark}]{\fancyplain{}{\bfseries\thepage}}
\cfoot{}
\linespread{1.3}

\begin{document}
\setlength{\headheight}{14.49998pt}
\addtolength{\topmargin}{-2.49998pt}

\begin{titlepage}
\thispagestyle{empty}
\topmargin=6.5cm
\raggedleft
\large
This is the \textsc{Dedication}:\\
you can write whatever you want here, \\
or nothing at all \ldots
\newpage
\clearpage{\pagestyle{empty}\cleardoublepage}
\end{titlepage}

\pagenumbering{roman}
\chapter*{Introduction}
\rhead[\fancyplain{}{\bfseries INTRODUCTION}]{\fancyplain{}{\bfseries\thepage}}
\lhead[\fancyplain{}{\bfseries\thepage}]{\fancyplain{}{\bfseries INTRODUCTION}}
\addcontentsline{toc}{chapter}{Introduction}

\subfile{sections/Intro/intro.tex}

\clearpage{\pagestyle{empty}\cleardoublepage}
\tableofcontents

% \clearpage{\pagestyle{empty}\cleardoublepage}
% \listoffigures

% \clearpage{\pagestyle{empty}\cleardoublepage}
% \listoftables

\clearpage{\pagestyle{empty}\cleardoublepage}

\chapter{Experimental Evaluation}
\rhead[\fancyplain{}{\bfseries EXPERIMENTAL EVALUATION}]{\fancyplain{}{\bfseries\thepage}}
\lhead[\fancyplain{}{\bfseries\thepage}]{\fancyplain{}{\bfseries\rightmark}}
\pagenumbering{arabic}

\subfile{sections/c1/c1.tex}

\chapter*{Appendix}
\rhead[\fancyplain{}{\bfseries APPENDIX}]{\fancyplain{}{\bfseries\thepage}}
\lhead[\fancyplain{}{\bfseries\thepage}]{\fancyplain{}{\bfseries APPENDIX}}
\addcontentsline{toc}{chapter}{Appendix}

\subfile{sections/appendix/appendix.tex}

\clearpage{\pagestyle{empty}\cleardoublepage}
\begin{thebibliography}{90}             
\rhead[\fancyplain{}{\bfseries BIBLIOGRAPHY}]{\fancyplain{}{\bfseries\thepage}}
\lhead[\fancyplain{}{\bfseries\thepage}]{\fancyplain{}{\bfseries BIBLIOGRAPHY}}


\addcontentsline{toc}{chapter}{Bibliography}

\bibitem{GeminiAPI} "Gemini API Docs and Reference," Google AI for Developers. https://ai.google.dev/gemini-api/docs (accessed Dec. 10, 2025). 
\bibitem{GeminiAPIver} "API versions explained," Google AI for Developers, 2025. https://ai.google.dev/gemini-api/docs/api-versions (accessed Dec. 11, 2025).
\bibitem{GroqAPI} "GroqCloud," Groq.com, 2024. https://console.groq.com/docs/overview (accessed Dec. 10, 2025).
\bibitem{GroqRL} "Rate Limits - GroqDocs," GroqDocs, 2025. https://console.groq.com/docs/rate-limits (accessed Dec. 10, 2025).
\bibitem{GeminiRL} "Rate limits," Google AI for Developers, 2025. https://ai.google.dev/gemini-api/docs/rate-limits (accessed Dec. 10, 2025).
\bibitem{ContextRot} Kelly Hong, Anton Troynikov, Jeff Huber, "Context Rot: How Increasing Input Tokens Impacts LLM Performance," Trychroma.com, 2025. https://research.trychroma.com/context-rot?ref=blog.promptlayer.com (accessed Dec. 11, 2025).
\bibitem{MznProblems} "MiniZinc - List of Problems and Globals used in the MiniZinc Challenge," Minizinc.org, 2025. https://www.minizinc.org/challenge/globals/ (accessed Dec. 11, 2025).
\bibitem{MznResults25} "MiniZinc - Challenge 2025 Results," Minizinc.org, 2025. https://www.minizinc.org/challenge/2025/results/ (accessed Dec. 11, 2025).
\bibitem{PhilMznChallenge} P. J. Stuckey, T. Feydy, A. Schutt, G. Tack, and J. Fischer, "The MiniZinc Challenge 2008-2013," AI Magazine, vol. 35, no. 2, p. 55, Jun. 2014, doi: https://doi.org/10.1609/aimag.v35i2.2539.
\bibitem{MiniZinc}N. Nethercote, P. J. Stuckey, R. Becket, S. Brand, G. J. Duck, and G. Tack, "MiniZinc: Towards a Standard CP Modelling Language," Springer eBooks, pp. 529-543, Oct. 2007, doi: https://doi.org/10.1007/978-3-540-74970-7\_38.
\bibitem{evaluationMetaSolvers} R. Amadini, Maurizio Gabbrielli, T. Liu, and J. Mauro, "On the Evaluation of (Meta-)solver Approaches," Journal of Artificial Intelligence Research, vol. 76, pp. 705-719, Mar. 2023, doi: https://doi.org/10.1613/jair.1.14102.
\bibitem{PortfolioAproaches} R. Amadini, Maurizio Gabbrielli, and J. Mauro, "Portfolio Approaches for Constraint Optimization Problems," Lecture notes in computer science, pp. 21-35, Jan. 2014, doi: https://doi.org/10.1007/978-3-319-09584-4\_3.
\bibitem{lindauer2019} Lindauer, M., van Rijn, J. N., \& Kotthoff, L. (2019). "The algorithm selection competitions" 2015 and 2017.Artificial Intelligence,272, 86-100.
\bibitem{LLMdistract} Shi, F., Chen, X., Misra, K., Scales, N., Dohan, D., Chi, E., Sch{\"a}rli, N., and Zhou, D. (2023). Large Language Models Can Be Easily Distracted by Irrelevant Context. arXiv preprint arXiv:2302.00093. doi: https://doi.org/10.48550/arXiv.2302.00093
\bibitem{RBFormatting}Sumanth P, "Agentic Prompt Engineering: A Deep Dive into LLM Roles and Role-Based Formatting," Clarifai.com, Jul. 2025. https://www.clarifai.com/blog/agentic-prompt-engineering\#title\_1 (accessed Jan. 03, 2026).
\bibitem{mzn2feat}R. Amadini, M. Gabbrielli, and J. Mauro. An Enhanced Features Extractor for a Portfolio of Constraint Solvers. In SAC, 2014.
\end{thebibliography}

\clearpage{\pagestyle{empty}\cleardoublepage}
\chapter*{Acknowledgements}
\thispagestyle{empty}
Here you can thank whoever you want.
\end{document}